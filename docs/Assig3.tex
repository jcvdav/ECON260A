\documentclass[]{article}
\usepackage{lmodern}
\usepackage{amssymb,amsmath}
\usepackage{ifxetex,ifluatex}
\usepackage{fixltx2e} % provides \textsubscript
\ifnum 0\ifxetex 1\fi\ifluatex 1\fi=0 % if pdftex
  \usepackage[T1]{fontenc}
  \usepackage[utf8]{inputenc}
\else % if luatex or xelatex
  \ifxetex
    \usepackage{mathspec}
  \else
    \usepackage{fontspec}
  \fi
  \defaultfontfeatures{Ligatures=TeX,Scale=MatchLowercase}
\fi
% use upquote if available, for straight quotes in verbatim environments
\IfFileExists{upquote.sty}{\usepackage{upquote}}{}
% use microtype if available
\IfFileExists{microtype.sty}{%
\usepackage{microtype}
\UseMicrotypeSet[protrusion]{basicmath} % disable protrusion for tt fonts
}{}
\usepackage[margin=1in]{geometry}
\usepackage{hyperref}
\hypersetup{unicode=true,
            pdftitle={Homework Challenge \#2},
            pdfauthor={Villaseñor-Derbez, J.C.},
            pdfborder={0 0 0},
            breaklinks=true}
\urlstyle{same}  % don't use monospace font for urls
\usepackage{longtable,booktabs}
\usepackage{graphicx,grffile}
\makeatletter
\def\maxwidth{\ifdim\Gin@nat@width>\linewidth\linewidth\else\Gin@nat@width\fi}
\def\maxheight{\ifdim\Gin@nat@height>\textheight\textheight\else\Gin@nat@height\fi}
\makeatother
% Scale images if necessary, so that they will not overflow the page
% margins by default, and it is still possible to overwrite the defaults
% using explicit options in \includegraphics[width, height, ...]{}
\setkeys{Gin}{width=\maxwidth,height=\maxheight,keepaspectratio}
\IfFileExists{parskip.sty}{%
\usepackage{parskip}
}{% else
\setlength{\parindent}{0pt}
\setlength{\parskip}{6pt plus 2pt minus 1pt}
}
\setlength{\emergencystretch}{3em}  % prevent overfull lines
\providecommand{\tightlist}{%
  \setlength{\itemsep}{0pt}\setlength{\parskip}{0pt}}
\setcounter{secnumdepth}{0}
% Redefines (sub)paragraphs to behave more like sections
\ifx\paragraph\undefined\else
\let\oldparagraph\paragraph
\renewcommand{\paragraph}[1]{\oldparagraph{#1}\mbox{}}
\fi
\ifx\subparagraph\undefined\else
\let\oldsubparagraph\subparagraph
\renewcommand{\subparagraph}[1]{\oldsubparagraph{#1}\mbox{}}
\fi

%%% Use protect on footnotes to avoid problems with footnotes in titles
\let\rmarkdownfootnote\footnote%
\def\footnote{\protect\rmarkdownfootnote}

%%% Change title format to be more compact
\usepackage{titling}

% Create subtitle command for use in maketitle
\newcommand{\subtitle}[1]{
  \posttitle{
    \begin{center}\large#1\end{center}
    }
}

\setlength{\droptitle}{-2em}

  \title{Homework Challenge \#2}
    \pretitle{\vspace{\droptitle}\centering\huge}
  \posttitle{\par}
  \subtitle{ECON 260A}
  \author{Villaseñor-Derbez, J.C.}
    \preauthor{\centering\large\emph}
  \postauthor{\par}
      \predate{\centering\large\emph}
  \postdate{\par}
    \date{2018-11-07}

\usepackage{float}
\floatplacement{figure}{H}

\begin{document}
\maketitle

\section{Set up}\label{set-up}

Let \(x_{it}\) be the stock of an invasive species in patch \(i\) at the
beginning of time period \(t\), and \(h_{it}\) is the control in patch
\(i\) during time period \(t\). The timing is as follows: The stock is
observed in each patch, some level of control is undertaken in each
patch, the remaining stock grows, and then moves across space. Movement
from patch \(i\) to patch \(j\) is given by the constant \(D_{ij}\). So
the equation of motion is:

\[
x_{it+1} = \sum_{j = 1}^N{D_{ji}g(e_{jt})}
\]

where \(e_{jt}\) is the residual stock in patch \(j\) and \(N\) is the
number of patches. If the stock at the beginning of the period in patch
\(i\) is \(x_i\) and the control is \(h_i\) (leaving residual stock
\(e_i\)), then the total control cost during that period in patch \(i\)
is \(\int_{e_i}^{x_i}{x_i e_i \theta_ic(s)\;ds}\), where the
downward-sloping function \(\theta_ic(s)\) is the marginal control cost
when the stock is \(s\) (the parameter \(\theta_i\) is a constant).
After control takes place, but before growth and spread occur, the
residual stock imposes a patch-specific marginal damage of \(k_i\), so
the total damage in patch \(i\) during period \(t\) is given by
\(k_ie_i\).

\section{Suppose each patch is owned by a separate landowner, and that
all landowners were myopic. Describe the dynamics and the steady state
of this
system.}\label{suppose-each-patch-is-owned-by-a-separate-landowner-and-that-all-landowners-were-myopic.-describe-the-dynamics-and-the-steady-state-of-this-system.}

\section{For the remainder of this assignment, assume that a central
planner can determine the level of control in each patch in each time
period}\label{for-the-remainder-of-this-assignment-assume-that-a-central-planner-can-determine-the-level-of-control-in-each-patch-in-each-time-period}

\subsection{\texorpdfstring{Period \(t\) dynamic programming
equation}{Period t dynamic programming equation}}\label{period-t-dynamic-programming-equation}

\subsection{Fixed-time, no salvage
value}\label{fixed-time-no-salvage-value}

\subsection{Backward induction}\label{backward-induction}

\subsection{\texorpdfstring{Dependence of control in \(i\) on
\(x, k, \phi D and D\)}{Dependence of control in i on x, k, \textbackslash{}phi D and D}}\label{dependence-of-control-in-i-on-x-k-phi-d-and-d}


\end{document}
